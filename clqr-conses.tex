% Copyright (C) 2008 Bert Burgemeister
%
% Permission is granted to copy, distribute and/or modify this
% document under the terms of the GNU Free Documentation License,
% Version 1.2 or any later version published by the Free Software
% Foundation; with no Invariant Sections, no Front-Cover Texts and
% no Back-Cover Texts. For details see file COPYING.
%

%%%%%%%%%%%%%%%%%%%%%%%%%%%%%%%%%%%%%%%%%%%%%%%%%%
\section{Conses} 
%%%%%%%%%%%%%%%%%%%%%%%%%%%%%%%%%%%%%%%%%%%%%%%%%%

%%%%%%%%%%%%%%%%%%%%%%%%%%%%%%%%%%%%%%%%%%%%%%%%%%
\subsection{Predicates} 
%%%%%%%%%%%%%%%%%%%%%%%%%%%%%%%%%%%%%%%%%%%%%%%%%%
\begin{LIST}{1cm}

  \IT{\arrGOO{(\FU*{CONSP} \VAR{ foo})\\
      (\FU*{LISTP} \VAR{ foo})}{.}}
  {
    Return \retval{\T} if \VAR{foo} is of indicated type.
    }

  \IT{(\FU*{ENDP} \VAR{list})}
  {
  Return \retval{\T} if \VAR{list} is \NIL.
  }

  \IT{(\FU*{NULL} \VAR{foo})}
  {Return \retval{\T} if \VAR{foo} is \NIL.
    }

  \IT{(\FU*{ATOM} \VAR{foo})}
  {Return \retval{\T} if \VAR{foo} is not a
  \kwd{cons}. 
  }

  \IT{(\FU*{TAILP} \VAR{foo} \VAR{list})}
  {
  Return \retval{\T} if \VAR{foo} is a tail of \VAR{list}.
  }

  \IT{(\FU*{MEMBER} \VAR{foo} \VAR{list}
    \orGOO{\kwd{:test} \VAR{ function}\\
      \kwd{:test-not} \VAR{ function}\\
      \kwd{:key} \VAR{ function}}{\}})}
  {
  Return \retval{tail of \VAR{list}} starting with
  its first element matching \VAR{foo}. Return \retval{\NIL} if
  there is no such element.
  }

  \IT{(\xorGOO{\FU*{MEMBER-IF}\\
      \FU*{MEMBER-IF-NOT}}{\}}
    \VAR{test} \VAR{list}
    \Op{\kwd{:key} \VAR{ function}})}
  {
  Return \retval{tail of \VAR{list}} starting with
  its first element satisfying \VAR{test}. Return \retval{\NIL} if
  there is no such element.
  }

  \IT{(\FU*{SUBSETP} \VAR{list-a} \VAR{list-b} 
    \orGOO{\kwd{:test} \VAR{ function}\\
      \kwd{:test-not} \VAR{ function}\\
      \kwd{:key} \VAR{ function}}{\}})}
  {
  Return \retval{\T} if \VAR{list-a} is a subset of
  \VAR{list-b}.
  }

\end{LIST}


%%%%%%%%%%%%%%%%%%%%%%%%%%%%%%%%%%%%%%%%%%%%%%%%%%
\subsection{Lists} 
%%%%%%%%%%%%%%%%%%%%%%%%%%%%%%%%%%%%%%%%%%%%%%%%%%
\begin{LIST}{1cm}

  \IT{(\FU*{CONS} \VAR{foo} \VAR{bar})}
  {
  Return new cons \retval{(\VAR{foo} \LIT{.} \VAR{bar})}.
  }

  \IT{(\FU*{LIST} \OPn{\VAR{foo}})}
  {Return \retval{list of \VAR{foo}s}.
    }

  \IT{(\FU*{LIST*} \RP{\VAR{foo}})}
  {Return \retval{list of \VAR{foo}s}
  with last \VAR{foo} becoming cdr of last cons. Return
  \retval{\VAR{foo}} if only one \VAR{foo} given.
  }

  \IT{(\FU*{MAKE-LIST} \VAR{num} \Op{\kwd{:initial-element}
      \VAR{foo}\DF{\NIL}})}
  {
    New \retval{list} with \VAR{num} elements set to \VAR{foo}.
  }

  \IT{(\FU*{LIST-LENGTH} \VAR{list})}
  {\retval{Length} of \VAR{list};
  \retval{\NIL} for circular \VAR{list}.
  }

  \IT{(\FU*{CAR} \VAR{ list})}
  {
  \retval{car of \VAR{list}} or \retval{\NIL} if \VAR{list} is
  \NIL. \kwd{SETF}able. 
  }

  \IT{\arrGOO{(\FU*{CDR} \VAR{ list})\\
      (\FU*{REST} \VAR{ list})}{.}}
  {
  \retval{cdr of \VAR{list}} or \retval{\NIL} if \VAR{list}
  is \NIL. \kwd{SETF}able. 
  }

  \IT{(\FU*{NTHCDR} \VAR{n list})}
  {Return \retval{tail of \VAR{list}} after calling \FU{cdr} \VAR{n} times.
    }

  \IT{(\Goo{\FU*{FIRST}\XOR\FU*{SECOND}\XOR\FU*{THIRD}\XOR\FU*{FOURTH}\XOR\FU*{FIFTH}\XOR\FU*{SIXTH}\XOR\dots\XOR\FU*{NINTH}\XOR\FU*{TENTH}}
    \VAR{list})}
  {
  \index{SEVENTH}%
  \index{EIGHTH}%
  Return \retval{nth element of \VAR{list}} if any,
  or \retval{\NIL} otherwise. \kwd{SETF}able.
  }

  \IT{(\FU*{NTH} \VAR{n list})}
  {Return \VAR{n}th element (zero-indexed) of \VAR{list}. \kwd{setf}able.
    }

  \IT{(\FU{C}\VAR{X}\kwd{R} \VAR{list})}
  {
  \index{CAAR}%
  \index{CADR}%
  \index{CDAR}%
  \index{CDDR}%
  With \VAR{X} being one to four
  \kwd{a}s and \kwd{d}s representing \FU{CAR}s and \FU{CDR}s, e.g.
  (\FU{CADR} \VAR{bar}) is equivalent to (\FU{CAR} (\FU{CDR}
  \VAR{bar})).
  \kwd{SETF}able.
  }

  \IT{(\FU*{LAST} \VAR{list} \Op{\VAR{num}\DF{1}})}
  {
  Return list of \retval{last \VAR{num}
    conses} of \VAR{list}.
  }

  \IT{(\xorGOO{\FU*{BUTLAST}\\
      \FU*{NBUTLAST}}{\}} \VAR{list}
    \Op{\VAR{num}})}
  {
  Return \retval{list} excluding last \VAR{num}
    conses of \VAR{list}. \VAR{list} is unmodified/possibly modified,
    respectively. 
  }

  \IT{(\xorGOO{\FU*{RPLACA}\\
    \FU*{RPLACD}}{\}} \VAR{cons} \VAR{object})}
  {
  Replace car, or cdr, respectively, of \retval{\VAR{cons}} with \VAR{object}.
  }

  \IT{(\FU*{LDIFF} \VAR{list} \VAR{foo})}
  {
  If \VAR{foo} is a tail of \VAR{list}, return \retval{preceding
    part of \VAR{list}}. Otherwise return \retval{\VAR{list}}.
  }

  \IT{(\FU*{ADJOIN} \VAR{foo} \VAR{list} \orGOO{\kwd{:test}
      \VAR{ function}\\
      \kwd{:test-not} \VAR{ function}\\
      \kwd{:key} \VAR{ function}}{\}})}
  {Return \retval{\VAR{list}} if \VAR{foo} is
  already member of \VAR{list}. If not, return \retval{(\FU{CONS}
    \VAR{foo} \VAR{list})}.
  }

  \IT{(\FU*{POP} \VAR{place})}
  {
  Set \VAR{place} to \retval{(\FU{CDR} \VAR{place})}, return
  \retval{(\FU{CAR} \VAR{place})}.
  }

  \IT{(\FU*{PUSH} \VAR{foo} \VAR{place})}
  {Set \VAR{place} to
  \retval{(\FU{cons} \VAR{foo} \VAR{place})}.
  }

  \IT{(\FU*{PUSHNEW} \VAR{foo} \VAR{place}  \orGOO{\kwd{:test}
      \VAR{ function}\\
      \kwd{:test-not} \VAR{ function}\\
      \kwd{:key} \VAR{ function}}{\}})}
  {Set \VAR{place} to
  \retval{(\FU{adjoin} \VAR{foo} \VAR{place})}.
  }

  \IT{\arrGOO{(\FU*{APPEND } \Op{\OPn{\VAR{list}} \VAR{ foo}})\\
      (\FU*{NCONC } \Op{\OPn{\VAR{list}} \VAR{ foo}})}{.}}
  {
  Return \retval{concatenated list}. \VAR{foo} can be of any type. \VAR{list}s are unmodified/possibly
  modified, respectively.
  }

  \IT{\arrGOO{(\FU*{REVAPPEND} \VAR{ list} \VAR{ foo})\\(\FU*{NRECONC}
      \VAR{ list} \VAR{ foo})}{.}}
  {
  Return \retval{concatenated list} after reversing order in
  \VAR{list}. \VAR{list} is unmodified/possibly
  modified, respectively.
  }

  \IT{(\xorGOO{\FU*{MAPCAR}\\
      \FU*{MAPLIST}}{\}} \VAR{function} \RP{\VAR{list}})}
  {
  Return \retval{list of return values} of \VAR{function} successively
  invoked with corresponding arguments, either cars or cdrs, respectively,
  from each \VAR{list}. 
  }

  \IT{(\xorGOO{\FU*{MAPCAN}\\
      \FU*{MAPCON}}{\}} \VAR{function} \RP{\VAR{list}})}
  {
  Return list of 
  \retval{concatenated return values} of
  \VAR{function} successively invoked with corresponding arguments,
  either cars or cdrs, respectively,
  from each \VAR{list}. \VAR{function} should return a list.
  }

  \IT{(\xorGOO{\FU*{MAPC}\\
      \FU*{MAPL}}{\}} \VAR{function} \RP{\VAR{list}})}
  {
  Return \retval{first \VAR{list}} after successively applying
  \VAR{function} to corresponding arguments, either cars or cdrs,
  respectively, from each \VAR{list}. \VAR{function} should have some side
  effects. 
  }

  \IT{(\FU*{COPY-LIST} \VAR{list})}
  {
  Return \retval{copy} of \VAR{list}.
  }

\end{LIST}


%%%%%%%%%%%%%%%%%%%%%%%%%%%%%%%%%%%%%%%%%%%%%%%%%%
\subsection{Association Lists} 
%%%%%%%%%%%%%%%%%%%%%%%%%%%%%%%%%%%%%%%%%%%%%%%%%%
\label{section:Association Lists}
\begin{LIST}{1cm}

  \IT{(\FU*{PAIRLIS} \VAR{keys} \VAR{values})}
  {
  Make \retval{association list} from lists \VAR{keys} and \VAR{values}.
  }

  \IT{(\FU*{ACONS} \VAR{key} \VAR{value} \VAR{a-list})}
  {
  Return \retval{\VAR{a-list}} with a (\VAR{key} . \VAR{value}) pair added.
  }

  \IT{\arrGOO{(\xorGOO{\FU*{ASSOC}\\
        \FU*{RASSOC}}{\}}
      \VAR{foo} \VAR{ a-list}
      \xorGOO{\kwd{:key} \VAR{ function}\\
        \kwd{:test} \VAR{ test}\\
        \kwd{:test-not} \VAR{ test}
      }{\}})\\
    (\xorGOO{\FU*{ASSOC-IF}\Op{\kwd{-NOT}}\\
      \FU*{RASSOC-IF}\Op{\kwd{-NOT}}}{\}} \VAR{test} \VAR{ a-list }
      \Op{\kwd{:key} \VAR{ function}})}{.}}
  {
\index{ASSOC-IF-NOT}\index{RASSOC-IF-NOT}%
  First \retval{cons} whose car, or cdr, respectively, satisfies \VAR{test}.
  }

  \IT{(\FU*{COPY-ALIST} \VAR{a-list})}
  {
  Return \retval{copy} of \VAR{a-list}.
  }

\end{LIST}


%%%%%%%%%%%%%%%%%%%%%%%%%%%%%%%%%%%%%%%%%%%%%%%%%%
\subsection{Trees} 
%%%%%%%%%%%%%%%%%%%%%%%%%%%%%%%%%%%%%%%%%%%%%%%%%%
\begin{LIST}{1cm}

  \IT{(\FU*{TREE-EQUAL} \VAR{foo} \VAR{bar} 
    \orGOO{\kwd{:test} \VAR{ function}\\
      \kwd{:test-not} \VAR{ function}}{\}})}
  {
  Return \retval{\T} if trees \VAR{foo} and \VAR{bar} have same
  shape and \kwd{eql} leaves.
  }

  \IT{(\xorGOO{\FU*{SUBST}\\
      \FU*{NSUBST}}{\}} \VAR{new} \VAR{old} \VAR{tree}
    \orGOO{\kwd{:key} \VAR{ function}\\
      \kwd{:test} \VAR{ function}\\
      \kwd{:test-not} \VAR{ function}}{\}})}
  {
  Make \retval{copy of
    \VAR{tree}} with each subtree or leaf matching \VAR{old} replaced by
  \VAR{new}. \VAR{tree} is unmodified/possibly modified, respectively.
  }

  \IT{(\xorGOO{\FU{SUBST-IF\Op{-NOT}}\\\FU{NSUBST-IF\Op{-NOT}}}{\}} 
    \VAR{new} \VAR{test} \VAR{tree}
    \Op{\kwd{:key} \VAR{function}})}
  {
  \index{SUBST-IF}%
  \index{SUBST-IF-NOT}%
  \index{NSUBST-IF}%
  \index{NSUBST-IF-NOT}%
  Make \retval{copy of
    \VAR{tree}} with each subtree or leaf
  satisfying test replaced by
  \VAR{new}. \VAR{sequence} is unmodified/possibly modified, respectively.
  }

  \IT{(\xorGOO{\FU*{SUBLIS}\\
      \FU*{NSUBLIS}}{\}} \VAR{a-list} \VAR{tree} \orGOO{%
      \kwd{:key} \VAR{ function}\\
      \kwd{:test} \VAR{ function}\\
      \kwd{:test-not} \VAR{ function}}{\}})}
  {
  Make \retval{copy of \VAR{tree}} with each subtree or leaf matching
  a key in \VAR{a-list} replaced by that key's value.
  \VAR{sequence} is unmodified/possibly modified, respectively.
  }

  \IT{(\FU*{COPY-TREE} \VAR{tree})} 
  {
    \retval{Copy of \VAR{tree}} with same shape and leaves.
  }

\end{LIST}


%%%%%%%%%%%%%%%%%%%%%%%%%%%%%%%%%%%%%%%%%%%%%%%%%%
\subsection{Sets} 
%%%%%%%%%%%%%%%%%%%%%%%%%%%%%%%%%%%%%%%%%%%%%%%%%%
\begin{LIST}{1cm}

  \IT{(\xorGOO{\Goo{\FU*{UNION}\XOR\FU*{NUNION}}\\
      \Goo{\FU*{INTERSECTION}\XOR\FU*{NINTERSECTION}}\\
      \Goo{\FU*{SET-DIFFERENCE}\XOR\FU*{NSET-DIFFERENCE}}\\
      \Goo{\FU*{SET-EXCLUSIVE-OR}\XOR\FU*{NSET-EXCLUSIVE-OR}}}{\}}
    \VAR{a} \VAR{b} 
    \orGOO{\kwd{:test} \VAR{ function}\\
      \kwd{:test-not} \VAR{ function}\\
      \kwd{:key} \VAR{ function}}{\}})}
  {
  Return \retval{$a\cup b$}, \retval{$a\cap b$}, \retval{$a\setminus b$}, or
  \retval{$a\,\triangle\, b$}, respectively, of lists \VAR{a} and \VAR{b}. Work
  non-destructively/destructively, respectively.
  }

\end{LIST}


